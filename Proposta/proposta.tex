\documentclass[conference]{IEEEtran}
\IEEEoverridecommandlockouts
% The preceding line is only needed to identify funding in the first footnote. If that is unneeded, please comment it out.
\usepackage{cite}
\usepackage{amsmath,amssymb,amsfonts}
\usepackage{algorithmic}
\usepackage{graphicx}
\usepackage{textcomp}
\usepackage{xcolor}
\usepackage{placeins}
\usepackage[hidelinks]{hyperref}
\usepackage{float}
\def\BibTeX{{\rm B\kern-.05em{\sc i\kern-.025em b}\kern-.08em
    T\kern-.1667em\lower.7ex\hbox{E}\kern-.125emX}}
\begin{document}

\title{Reconhecimento de Dígitos}

\author{\IEEEauthorblockN{Arthur Abrahão Santos Barbosa}
\IEEEauthorblockA{\textit{Universidade Federal de Pernambuco} \\
\textit{Centro de Informática}\\
Pernambuco, Brasil \\
aasb2@cin.ufpe.br}
\and
\IEEEauthorblockN{Filipe Samuel da Silva}
\IEEEauthorblockA{\textit{Universidade Federal de Pernambuco} \\
\textit{Centro de Informática}\\
Pernambuco, Brasil \\
fss8@cin.ufpe.br}

}

\maketitle





\section{Objetivos}
\subsection{Objetivo Geral}
Através de métodos de aprendizagem de máquina a partir de redes neurais, usando as abordagens de aprendizagem profunda (Deep Learning). Criar um modelo classificador para reconhecer, a partir uma imagem de entrada, os dígitos contidos nela.
\subsection{Objetivos Específicos}
\begin{itemize}
\item  Explicar o funcionamento de um modelo classificador de uma rede neural profunda
\item  Desenvolver um modelo de classificação de dígitos a partir de uma rede profunda convolucional
\item  Realizar testes sobre a performance do modelo
\end{itemize}
\section{Justificativa}

O problema de reconhecimento de dígitos por um dispositivo computacional a partir de uma imagem é em grande parte, por conta da dificuldade de se reconhecer os diferentes padrões que uma imagem que representa o objeto real pode assumir.
Para isso é necessário desenvolver um modelo de aprendizagem de máquina que reconheça as diferentes formas de representação desse valor numérico, independente de possíveis falhas na escrita(ex: uma folha de papel com um risco indesejado) ou na captura das imagens por uma câmera ou scanner(ex: uma imagem com um borrão/ pouca luminosidade).
Para reconhecer dígitos a partir de uma imagem, é necessário a implementação de uma rede neural convulucional utilizando segmentação. 

\section{Metodologia}

A partir de uma base de dados que contém imagens de digitos manuscritos, isto é, através de separação entre dados utilizados para os experimentos e dados para o treinamento do classificador e analisar quantitativamente as informações contidas nos campos da base de dados, isto é, fazer uma análise exploratória desses dados. 

Fazer o treinamento do Classificador de Digitos, e fazer uma validação. O projeto será dividido nas seguintes etapas:


\begin{itemize}
\item \textbf{Pesquisa sobre o tema:}
Através da pesquisa bibliográfica, estudar  a relevância do assunto, e suas aplicações.


\item \textbf{Base de dados:}
A base de dados está disponível em \cite{b1}. Se refere a um conjunto de Imagens de digitos de 0 a 9 manuscritos.


\item \textbf{Tratamento dos dados:}
Fazer a limpeza e seleção dos dados que serão usados no projeto. Os dados selecionados serão divididos em dois grupos, dados para treinamento do classificador e dados para o experimento.


\item \textbf{Analise Exploratória:}
Através do Uso da biblioteca pandas, numpy e matplotlib, analisar o comportamento e concentração dos tons de cinza em cada pixel

\item \textbf{Classificador de Digitos:}
	A criação do Classificador será dividida em duas partes, ambos serão implementados usando uma rede Neural  convolucional. Na Primeira parte será usado o dataset MNIST \cite{b1} e será criado um classificador que reconhece imagens que contém apenas um digito, Na segunda Parte será criado um classificador que reconhece mais de um digito na mesma imagem através do processo de segmentação e além disso os Caracteres '(', ')', '+', '-', 'x' e '/':
	
	
    \begin{itemize}
        \item \textbf{Classificador de Apenas Um Digito:}
        
        \begin{itemize}       
        \item \textbf{Estrutura do Modelo da Rede:}
        \end{itemize}
        
        \item \textbf{Classificador Multidigito:}
        
        \begin{itemize}       
        \item \textbf{Estrutura do Modelo da Rede:} Similarmenente ao modelo anterior, mas com algumas alterações para permitir que a imagem seja segmentada e reconheça mais de um digito por imagem.
        \end{itemize}
        
    \end{itemize}
    

    
\item \textbf{Experimentos:}
Com o classificador em mãos, realizar alguns experimentos e verificar seus resultados.
\item \textbf{Análise dos resultados:}
A partir dos dados obtidos nas etapas anteriores, analisar os resultados obtidos.

\item \textbf{Interface do Usuário (Opcional): } 
    Caso Haja tempo o suficiente planeja-se usar o classificador Multidigito para criar um simples problema que resolve cálculos matemáticos manuscrito que contenham apenas ás quatro operações básicas, tirando uma foto da folha de papel que os contém
\end{itemize}

\clearpage
\section*{Cronograma de Atividades}
\begin{table}[!ht]
	\centering
    \begin{small}
        \begin{tabular}{cc}
        	\\
        	%\multicolumn{2}{c}{{\fontsize{13}{\baselineskip} \selectfont C}{\fontsize{11}{\baselineskip}\selectfont RONOGRAMA DE}{\fontsize{13}{\baselineskip} \selectfont A}{\fontsize{11}{\baselineskip}\selectfont TIVIDADES }}\\ 
        	\\
            \hline
            Data                    & Atividades\\
            \hline
           
            06/11/21                & Pesquisa Bibliográfica e Escrever Relatório\\
            07/11/21                & Implementar Classificador Para um Único Dígito \\
            08/11/21                & Implementar Classificador Para um Único Dígito\\
            09/11/21                & Teste do Classificador e Análise Exploratória dos dados\\
            10/11/21                & Experimentos Escrever Resultados no Relatório\\
            11/11/21                & Implementar o Classificador Para Mais de um Digito\\
            12/11/21                & Implementar o Classificador Para Mais de um Digito\\
            13/11/21                & Implementar o Classificador Para Mais de um Digito e Teste do Classificador\\
            14/11/21                & Análise Exploratória dos Dados, Experimentos  e Escrever Resultados no Relatório \\
            15/11/21                & Consertar Bugs e Implementar Interface de Usuário\\
            16/11/21                & Escrever Relatório e Gravar a Apresentação\\
            17/11/21                & Entrega do Projeto\\
            
            \hline
        \end{tabular}
    \end{small}
\end{table}





\bibliography{mybib}
\nocite{*}
\bibliographystyle{IEEEtran}
\end{document}

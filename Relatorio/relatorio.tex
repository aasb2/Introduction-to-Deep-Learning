\documentclass[conference]{IEEEtran}
\IEEEoverridecommandlockouts
% The preceding line is only needed to identify funding in the first footnote. If that is unneeded, please comment it out.
\usepackage{cite}
\usepackage{amsmath,amssymb,amsfonts}
\usepackage{algorithmic}
\usepackage{graphicx}
\usepackage{textcomp}
\usepackage{xcolor}
\usepackage{placeins}
\usepackage[hidelinks]{hyperref}
\usepackage{float}
\def\BibTeX{{\rm B\kern-.05em{\sc i\kern-.025em b}\kern-.08em
    T\kern-.1667em\lower.7ex\hbox{E}\kern-.125emX}}
\begin{document}

\title{Redes Neurais Convolucionais}

\author{\IEEEauthorblockN{Arthur Abrahão Santos Barbosa}
\IEEEauthorblockA{\textit{Universidade Federal de Pernambuco} \\
\textit{Centro de Informática}\\
Pernambuco, Brasil \\
aasb2@cin.ufpe.br}
\and
\IEEEauthorblockN{Filipe Samuel da Silva}
\IEEEauthorblockA{\textit{Universidade Federal de Pernambuco} \\
\textit{Centro de Informática}\\
Pernambuco, Brasil \\
fss8@cin.ufpe.br}

}

\maketitle





\section{Objetivos}

\subsection{Objetivo Geral}

Desnvolver um Classificador Multiclasse que reconheça as imagens do dataset CIFAR100 \cite{dataset}. 
\subsection{Objetivos Específicos}
\begin{itemize}
\item Compreender a implementação de uma Rede Neural Convolucional
\item Demonstrar a Importância do Aprendizado Profundo e suas aplicações
\item Demonstrar a eficiência de três arquiteturas importantes para a história do Deep Learning
\end{itemize}
\section{Justificativa}
Este projeto foi escolhido com base no fato deste dataset ser bastante usado para testar redes neurais com imagens coloridas, 
e pelo fato de ter uma divisão bastante equilibrada dos dados.\cite{dataset}.


Sua função é verificar a qual das classes pertence uma imagem de tamanho 32x32.

\section{Base de Dados}





\section{Análise Exploratória dos Dados}
\subsection{Descrição Estatística dos dados}



\subsection{Sobre o Projeto}
%Para montar o classificador foi necessário passar  pelas seguintes etapas:



\section{Sobre as Métricas Utilizadas}

\subsection{Precision}

Precision é a razão
    \begin{equation}
        \frac{A_c}{A_c + A_e}
    \end{equation}

    onde:

    \begin{itemize}
    \item $A_c$ é o número de amostras corretamente classificadas de uma determinada classe.
    \item $A_e$ é o número de amostras  erroneamente classificadas como sendo desta determinada classe.
    \end{itemize}

    Precision é intuitivamente a habilidade do classificador não marcar como pertencente a uma classe uma amostra que não pertence a esta. O melhor valor de Precision é 1 e o pior é zero.
    \cite{b7}
\subsection{Accuracy} 

Accuracy é a fração de amostras preditas corretamente, e é dada pela seguinte fórmula:

\begin{equation}
    \frac{\sum\limits_{1}^{n}A_c}{\sum\limits_{1}^{n}A_t}
\end{equation}




onde:
\begin{itemize}
    \item $n$ é o número de classes
    \item $A_c$ é o número de amostras corretamente classificadas de uma determinada classe.
    \item $A_t$ é o número de amostras que pertencem a uma determinada classe
\end{itemize}

\cite{b8}
\subsection{Recall-Score}
O Recall Score é a razão:

    \begin{equation}
        \frac{A_c}{A_t}
    \end{equation}
    onde:
    \begin{itemize}
        \item $A_c$ é o número de amostras classificadas corretamente de uma determinada classe
        \item $A_t$ é o número de amostras que pertencem a esta classe
    \end{itemize}
    O Recall Score é intuitivamente a habilidade do classificador de encontrar todas as amostras pertencentes a uma classe especifica. O melhor valor do Recall Score é 1 e o pior valor é 0.
    \cite{b6}

\subsection{F1-Score}
O F1 Score pode ser interpretado como a média ponderada da precisão e recall. O melhor valor que o  F1 score pode alcançar é 1, o pior é 0. A contribuição relativa da precisão e recall para o F1 score são iguais. A fórmula para o F1 score é:
\begin{equation}
    F1 = \frac{2\cdot(precision\cdot recall)}{precision + recall}
\end{equation}
\cite{b2}
\subsection{Confusion Matrix}
No caso de classificação multiclasse, uma confusion matrix é dividida em NxN categorias(onde N é o número de classes do problema), cada uma apresentando a quantidade de amostras que se encaixam nesta.
A diagonal do meio representa a quantidade de amostras classificadas corretamente e as demais seções da matriz demonstram o número de amostras classificados erroneamente, a quais classes eles pertencem e em quais classes eles foram classificados.

\cite{metrics}




\section{Redes Neurais Convolucionais}


\subsection{AlexNet}

\subsection{GoogLeNet}

\subsection{Squeezenet}


\section{Experimentos}

\subsection{Experimentos Iniciais}

\subsection{CIFAR10}


\subsection{Similaridade de Pixel}

\subsection{Treinando os Modelos}

\subsection{Imagens em Grayscale}

\subsection{Sem Normalização e sem Data Augmentation}

\subsection{Apenas um Epoch}

\subsection{Limitando os Dados}

\subsection{Modelos Pré-Treinados}

\section{Análise dos Resultados}




\section{Conclusões e Discussões}


\bibliography{mybib}
\nocite{*}
\bibliographystyle{IEEEtran}
\end{document}

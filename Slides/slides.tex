\documentclass{beamer}
\usepackage[utf8]{inputenc}
% make pause work in aligned environment
\makeatletter
\renewrobustcmd{\beamer@@pause}[1][]{%
  \unless\ifmeasuring@%
  \ifblank{#1}%
    {\stepcounter{beamerpauses}}%
    {\setcounter{beamerpauses}{#1}}%
  \onslide<\value{beamerpauses}->\relax%
  \fi%
}
\makeatother

\title{Redes Neurais Convolucionais}
\usetheme{Madrid}
\setbeameroption{hide notes}

\begin{document}
\maketitle

\begin{frame}[allowframebreaks]
\frametitle{Overview}
\tableofcontents
\end{frame}

\begin{frame}
\section{Base de Dados}
\frametitle{Base de Dados}
\subsection{CIFAR100}
\framesubtitle{CIFAR100}

\begin{itemize}
    \item \textbf{aquatic mammals:} 	beaver, dolphin, otter, seal, whale
    \item \textbf{fish:} 	aquarium fish, flatfish, ray, shark, trout
    \item \textbf{flowers:} 	orchids, poppies, roses, sunflowers, tulips
    \item \textbf{food containers:} 	bottles, bowls, cans, cups, plates
    \item \textbf{fruit and vegetables:} 	apples, mushrooms, oranges, pears, sweet peppers
    \item \textbf{household electrical devices:} 	clock, computer keyboard, lamp, telephone, television
    \item \textbf{household furniture:} 	bed, chair, couch, table, wardrobe
    \item \textbf{insects:} 	bee, beetle, butterfly, caterpillar, cockroach
    \item \textbf{large carnivores:} 	bear, leopard, lion, tiger, wolf
    \item \textbf{large man-made outdoor things:} 	bridge, castle, house, road, skyscraper
    \item \textbf{large natural outdoor scenes:} 	cloud, forest, mountain, plain, sea
    \item \textbf{large omnivores and herbivores:} 	camel, cattle, chimpanzee, elephant, kangaroo
    \item \textbf{medium-sized mammals:} 	fox, porcupine, possum, raccoon, skunk

\end{itemize}

\end{frame}


\begin{frame}
\begin{itemize}
    \item \textbf{medium-sized mammals:} 	fox, porcupine, possum, raccoon, skunk
    \item \textbf{non-insect invertebrates:} 	crab, lobster, snail, spider, worm
    \item \textbf{people:} 	baby, boy, girl, man, woman
    \item \textbf{reptiles:} 	crocodile, dinosaur, lizard, snake, turtle
    \item \textbf{small mammals:} 	hamster, mouse, rabbit, shrew, squirrel
    \item \textbf{trees:} 	maple, oak, palm, pine, willow
    \item \textbf{vehicles 1:} 	bicycle, bus, motorcycle, pickup truck, train
    \item \textbf{vehicles 2:}	lawn-mower, rocket, streetcar, tank, tractor

\end{itemize}  
\end{frame}

\begin{frame}
\section{Análise Exploratória dos Dados}
\frametitle{Análise Exploratória dos Dados}
\begin{itemize}
\item Quantidades de Imagens por Classes
\item Imagem Média

\end{itemize}

\end{frame}
    






\begin{frame}
    \section{Redes Neurais Convolucionais}
    \frametitle{Redes Neurais Convolucionais}
        \begin{itemize}
            \item O Nosso Modelo
            \item Lenet
            \item AlexNet
        \end{itemize}
    
\end{frame}


\begin{frame}
\section{Experimentos} 
\frametitle{Experimentos}
\subsection{Métricas}
\framesubtitle{Métricas}
\begin{itemize}
\item Precision:
\begin{equation}
    \frac{A_c}{A_c + A_e}
\end{equation}
\item Accuracy
\begin{equation}
    \frac{\sum\limits_{1}^{n}A_c}{\sum\limits_{1}^{n}A_t}
\end{equation}
\item Recall Score
\begin{equation}
    \frac{A_c}{A_t}
\end{equation}
\item F1 Score
\begin{equation}
    \frac{2\cdot(precision\cdot recall)}{precision + recall}
\end{equation}
\end{itemize}
\end{frame}





\begin{frame}
\frametitle{Experimentos}
\begin{itemize}
    \item CIFAR10
    \item Treinando os Modelos
    \item Apenas um Epoch
    \item Limitando os Dados
    \item Sem Normalização
    \item LeNet com ReLu
\end{itemize}
\end{frame}















\begin{frame}
	\frametitle{Análise dos Resultados}
    \begin{itemize}
        \item{Comparação Entre os Modelos}
        \item Conclusões e Discussões
    \end{itemize}
 
\end{frame}
\end{document}

\documentclass{beamer}
\usepackage[utf8]{inputenc}
% make pause work in aligned environment
\makeatletter
\renewrobustcmd{\beamer@@pause}[1][]{%
  \unless\ifmeasuring@%
  \ifblank{#1}%
    {\stepcounter{beamerpauses}}%
    {\setcounter{beamerpauses}{#1}}%
  \onslide<\value{beamerpauses}->\relax%
  \fi%
}
\makeatother

\title{Redes Neurais Convolucionais}
\usetheme{Madrid}
\setbeameroption{hide notes}

\begin{document}
\maketitle

\begin{frame}[allowframebreaks]
\frametitle{Overview}
\tableofcontents
\end{frame}

\begin{frame}
\section{Base de Dados}
\frametitle{Base de Dados}
\subsection{CIFAR100}
\framesubtitle{CIFAR100}

\begin{itemize}
    \item \textbf{aquatic mammals:} 	beaver, dolphin, otter, seal, whale
    \item \textbf{fish:} 	aquarium fish, flatfish, ray, shark, trout
    \item \textbf{flowers:} 	orchids, poppies, roses, sunflowers, tulips
    \item \textbf{food containers:} 	bottles, bowls, cans, cups, plates
    \item \textbf{fruit and vegetables:} 	apples, mushrooms, oranges, pears, sweet peppers
    \item \textbf{household electrical devices:} 	clock, computer keyboard, lamp, telephone, television
    \item \textbf{household furniture:} 	bed, chair, couch, table, wardrobe
    \item \textbf{insects:} 	bee, beetle, butterfly, caterpillar, cockroach
    \item \textbf{large carnivores:} 	bear, leopard, lion, tiger, wolf
    \item \textbf{large man-made outdoor things:} 	bridge, castle, house, road, skyscraper
    \item \textbf{large natural outdoor scenes:} 	cloud, forest, mountain, plain, sea
    \item \textbf{large omnivores and herbivores:} 	camel, cattle, chimpanzee, elephant, kangaroo
    \item \textbf{medium-sized mammals:} 	fox, porcupine, possum, raccoon, skunk

\end{itemize}

\end{frame}


\begin{frame}
\begin{itemize}
    \item \textbf{medium-sized mammals:} 	fox, porcupine, possum, raccoon, skunk
    \item \textbf{non-insect invertebrates:} 	crab, lobster, snail, spider, worm
    \item \textbf{people:} 	baby, boy, girl, man, woman
    \item \textbf{reptiles:} 	crocodile, dinosaur, lizard, snake, turtle
    \item \textbf{small mammals:} 	hamster, mouse, rabbit, shrew, squirrel
    \item \textbf{trees:} 	maple, oak, palm, pine, willow
    \item \textbf{vehicles 1:} 	bicycle, bus, motorcycle, pickup truck, train
    \item \textbf{vehicles 2:}	lawn-mower, rocket, streetcar, tank, tractor

\end{itemize}  
\end{frame}

\begin{frame}
\section{Análise Exploratória dos Dados}
\frametitle{Análise Exploratória dos Dados}
\subsection{Quantidade de Imagens por Classe}
\framesubtitle{Quantidade de Imagens por Classe} 

\end{frame}
    
\begin{frame}
\frametitle{Análise Exploratória dos Dados}
\subsection{Imagens Médias}
\framesubtitle{Imagens Médias}

\end{frame}


\begin{frame}
\frametitle{Análise Exploratória dos Dados}
\framesubtitle{Gráficos}



      
\end{frame}


\begin{frame}
    \frametitle{Análise Exploratória dos Dados}
    \framesubtitle{Gráficos}
    
          
\end{frame}
    
\begin{frame}
    \frametitle{Análise Exploratória dos Dados}
    \framesubtitle{Gráficos}
    
    
          
\end{frame}

\begin{frame}
    \section{Classificador Ingênuo de Bayes}
    \frametitle{Classificador Ingênuo de Bayes}
    \subsection{História}
    \framesubtitle{O que é Naive Bayes}
    
    Baseado no Teorema de Bayes, nome em homenagem ao matemático e pastor presbiteriano inglês Thomas Bayes, que formulou uma função probabilística com o ideal de provar a existência de Deus, é um algoritmo de classificação probabilística muito utilizado para aprendizado de máquina (Machine Learning).
    
\end{frame}

\begin{frame}
\frametitle{Classificador Ingênuo de Bayes}
\subsection{Definição Formal do Teorema de Bayes}
\framesubtitle{Definição Formal do Teorema de Bayes}
\begin{equation}
    P(A|B) = \frac{P(B|A)P(A)}{P(B)}
\end{equation}

\begin{itemize}
\item $P(A|B)$ : Probabilidade do evento A ocorrer dado que o evento B ocorreu.
\item $P(B|A)$ : Probabilidade do evento B ocorrer dado que o evento A ocorreu.
\item $P(A)$   : Probabilidade do evento A ocorrer
\item $P(B)$   : Probabilidade do evento B ocorrer. 
\end{itemize}

\end{frame}

\begin{frame}
\frametitle{Classificador Ingênuo de Bayes}
\subsection{Tipos de Classificadores Ingênuo de Bayes}
\framesubtitle{Tipos de Classificadores Ingênuo de Bayes}
\begin{itemize}
\item Bayes Ingênuo Gaussiano
\item Bayes Ingênuo Categórico
\end{itemize}
\end{frame}

\begin{frame}
    \frametitle{Classificador Ingênuo de Bayes}
    \subsection{Vantagens}
    \framesubtitle{Vantagens}
    \begin{itemize}
    \item Rápido
    \item Eficiente
    \item Lida com múltiplos tidos de dados
    \item Ignora características irrelevantes
    
    \end{itemize}
    \end{frame}
    \begin{frame}
    \frametitle{Classificador Ingênuo de Bayes}
    \subsection{Desvantagens}
    \framesubtitle{Desvantagens}
    \begin{itemize}
    \item Previsão falha em frequência zero
    \item Ignorar a correlação entre as variáveis
    \end{itemize}
    \end{frame}

\begin{frame}
\frametitle{Classificador Ingênuo de Bayes}
\subsection{Sobre o Projeto}
\framesubtitle{Sobre o Projeto}
 
\end{frame}


\begin{frame}
\section{Experimentos} 
\frametitle{Experimentos}
\subsection{Experimentos Iniciais}
\framesubtitle{Experimentos Iniciais}
\begin{itemize}
\item Precision:
\begin{equation}
    \frac{t_p}{t_p + f_p}
\end{equation}
\item Accuracy
\begin{equation}
    \frac{t_p + t_n}{t_p + t_n + f_p + f_n}
\end{equation}
\item Recall Score
\begin{equation}
    \frac{t_p}{t_p + f_n}
\end{equation}
\item F1 Score
\begin{equation}
    \frac{2\cdot(precision\cdot recall)}{precision +  recall}
\end{equation}
\end{itemize}
\end{frame}





\begin{frame}
\frametitle{Experimentos}
\framesubtitle{Experimentos Iniciais}
\begin{table}[H]
    \centering
    \begin{small}
        \begin{tabular}{ccc}
            \\
            %\multicolumn{2}{c}{{\fontsize{13}{\baselineskip} \selectfont C}{\fontsize{11}{\baselineskip}\selectfont RONOGRAMA DE}{\fontsize{13}{\baselineskip} \selectfont A}{\fontsize{11}{\baselineskip}\selectfont TIVIDADES }}\\ 
            \\
            \hline
                                    & Categórico       & Gaussiano\\
            \hline
            Precision               & 0.89             & 0.84\\
            Accuracy                & 0.89             & 0.84\\
            Recall Score            & 0.89             & 0.84\\
            F1-Score                & 0.89             & 0.84\\
            
            \hline
        \end{tabular}
    \end{small}
\end{table}
    \begin{columns}
        \begin{column}{0.5\textwidth}
            \begin{table}[H]

                \centering
                \caption{\label{tab:cr1-cnb} Relatório de Classificação por Label do Classificador Categórico.}
                \begin{small}
                    \begin{tabular}{ccc}
                    
                        \hline
                                                & 0                & 1\\
                        \hline
                        Precision               & 0.93             & 0.53\\
                        Recall Score            & 0.95             & 0.43\\
                        F1-Score                & 0.94             & 0.47\\
                        
                        \hline
                    \end{tabular}
                \end{small}
            
            \end{table}
        \end{column}
        \begin{column}{0.5\textwidth}
            \begin{table}[H]

                \centering
                \caption{\label{tab:cr1-gnb} Relatório de Classificação por Label do Classificador Gaussiano}
                \begin{small}
                    \begin{tabular}{ccc}
                    
                        \hline
                                                & 0                & 1\\
                        \hline
                        Precision                & 0.89             & 0.49\\
                        Recall Score            & 1.00             & 0.03\\
                        F1-Score                & 0.94             & 0.06\\
                        
                        \hline
                    \end{tabular}
                \end{small}
            
            \end{table}
        \end{column}
        \end{columns}    
\end{frame}

\begin{frame}
    \frametitle{Experimentos}
    \framesubtitle{Experimentos Iniciais}
    
\end{frame}







\begin{frame}
    \frametitle{Experimentos}
    \subsection{Usando Apenas a Variável Age para Treino}
    \framesubtitle{Usando Apenas a Variável Age para Treino}
    \begin{table}[H]
        \centering
        \begin{small}
            \begin{tabular}{ccc}
                \\
                %\multicolumn{2}{c}{{\fontsize{13}{\baselineskip} \selectfont C}{\fontsize{11}{\baselineskip}\selectfont RONOGRAMA DE}{\fontsize{13}{\baselineskip} \selectfont A}{\fontsize{11}{\baselineskip}\selectfont TIVIDADES }}\\ 
                \\
                \hline
                                        & Categórico       & Gaussiano\\
                \hline
                Precision               & 0.88             & 0.88\\
                Accuracy                & 0.88             & 0.88\\
                Recall Score            & 0.88             & 0.88\\
                F1-Score                & 0.88             & 0.88\\
                
                \hline
            \end{tabular}
        \end{small}
    \end{table}
        \begin{columns}
            \begin{column}{0.5\textwidth}
                \begin{table}[H]

                    \centering
                    \caption{\label{tab:cr2-cnb} Relatório de Classificação por Label do Classificador Categórico.}
                    \begin{small}
                        \begin{tabular}{ccc}
                        
                            \hline
                                                    & 0                & 1\\
                            \hline
                            Precision               & 0.88             & 0.50\\
                            Recall Score            & 1.00             & 0.02\\
                            F1-Score                & 0.94             & 0.04\\
                            
                            \hline
                        \end{tabular}
                    \end{small}
                \end{table}
            \end{column}
            \begin{column}{0.5\textwidth}
                \begin{table}[H]

                    \centering
                    \caption{\label{tab:cr2-gnb} Relatório de Classificação por Label do Classificador Gaussiano}
                    \begin{small}
                        \begin{tabular}{ccc}
                        
                            \hline
                                                    & 0                & 1\\
                            \hline
                            Precision               & 0.88             & 0.48\\
                            Recall Score            & 1.00             & 0.03\\
                            F1-Score                & 0.94             & 0.05\\
                            
                            \hline
                        \end{tabular}
                    \end{small}
                
                \end{table}
            \end{column}
            \end{columns}    
    \end{frame}
    
\begin{frame}
    \frametitle{Experimentos}
    \framesubtitle{Usando Apenas a Variável Age para Treino}
    
\end{frame}





\begin{frame}
    \frametitle{Experimentos}
    \subsection{Usando Apenas Variáveis Numéricas Para Treino}
    \framesubtitle{Usando Apenas Variáveis Numéricas Para Treino}
    \begin{table}[H]
        \centering
        \caption{\label{tab:cr3-gt} Comparação entre o Classificador Categórico e o Gaussiano}
        \begin{small}
            \begin{tabular}{ccc}
                \\
                %\multicolumn{2}{c}{{\fontsize{13}{\baselineskip} \selectfont C}{\fontsize{11}{\baselineskip}\selectfont RONOGRAMA DE}{\fontsize{13}{\baselineskip} \selectfont A}{\fontsize{11}{\baselineskip}\selectfont TIVIDADES }}\\ 
                \\
                \hline
                                        & Categórico       & Gaussiano\\
                \hline
                Precision               & 0.89             & 0.89\\
                Accuracy                & 0.89             & 0.89\\
                Recall Score            & 0.89             & 0.89\\
                F1-Score                & 0.89             & 0.89\\
                
                \hline
            \end{tabular}
        \end{small}
    \end{table}


        \begin{columns}
            \begin{column}{0.5\textwidth}
                \begin{table}[H]
                    \caption{\label{tab:cr3-cnb} Relatório de Classificação por Label do Classificador Categórico.}
                    \centering
                    \begin{small}
                        \begin{tabular}{ccc}
                        
                            \hline
                                                    & 0                & 1\\
                            \hline
                            Precision               & 0.89             & 0.63\\
                            Recall Score            & 0.99             & 0.11\\
                            F1-Score                & 0.94             & 0.18\\
                            
                            \hline
                        \end{tabular}
                    \end{small}
                \end{table}
            \end{column}
            \begin{column}{0.5\textwidth}
                \begin{table}[H]

                    \centering
                    \caption{\label{tab:cr3-gnb} Relatório de Classificação por Label do Classificador Gaussiano}
                    \begin{small}
                        \begin{tabular}{ccc}
                        
                            \hline
                                                    & 0                & 1\\
                            \hline
                            Precision               & 0.91             & 0.53\\
                            Recall Score            & 0.96             & 0.32\\
                            F1-Score                & 0.94             & 0.40\\
                            
                            \hline
                        \end{tabular}
                    \end{small}
                
                \end{table}
            \end{column}
            \end{columns}    
    \end{frame}
    
\begin{frame}
    \frametitle{Experimentos}
    \framesubtitle{Usando Apenas Variáveis Numéricas Para Treino}
    
\end{frame}

\begin{frame}
\section{Análise dos Resultados}
\subsection{Resultados Iníciais}
\frametitle{Análise dos Resultados}
\framesubtitle{Resultados Iniciais}  
\begin{itemize}
\item Classificador Gaussiano vs Classificador Categórico
\item Número de Features
\item Falsos positivos vs Falsos Negativos
\end{itemize}
\end{frame}

\begin{frame}
    \frametitle{Análise dos Resultados}  
    \subsection{Perfil Mais Receptivo}
    \framesubtitle{Perfil Mais Receptivo}
    \begin{itemize}
    \item Profissão: estudante
    \item Estado Civil: divorciado
    \item Credito Pessoal: possui
    \item Credito de Habitação: não possui
    \item Tipo de Contato: celular
    \item Educação: ensino médio
    \item Mês da Campanha: setembro
    \end{itemize}
\end{frame}

\begin{frame}
    \frametitle{Análise dos Resultados}  
    \subsection{Perfil Menos Receptivo}
    \framesubtitle{Perfil Menos Receptivo}
    \begin{itemize}
    \item Profissão: operario
    \item Estado Civil: casado
    \item Credito Pessoal: não possui
    \item Credito de Habitação: possui
    \item Tipo de Contato: unknown
    \item Educação: ensino fundamental
    \item Mês da Campanha: maio
    \end{itemize}
\end{frame}


\begin{frame}
	\frametitle{Análise dos Resultados}
    \subsection{Categorias exóticas}
	\framesubtitle{Categorias exóticas}
 
\end{frame}
\end{document}
